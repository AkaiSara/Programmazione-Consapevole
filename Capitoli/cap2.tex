\chapter{Tipi predefiniti e variabili}
\noindent Come (quasi) tutti i linguaggi di programmazione, anche il C++ \textit{offre} dei tipi belli e pronti per l'uso.
Si chiamano tipi \textbf{predefiniti} e naturalmente servono a rappresentare valori praticamente onnipresenti nei problemi che si affrontano con la programmazione e cioè gli interi, i reali, i caratteri, e i booleani.
Cercheremo di presentarli in modo particolarmente compatto e mettendo in luce immediatamente i problemi che nascono dalla coesistenza nei nostri programmi di valori di tipo diverso e che di conseguenza sono rappresentati nella RAM in modo diverso.
In particolare, discuteremo di conversioni tra valori di tipi diversi e accenneremo anche al concetto di \textit{sovraccaricamento o overloading} degli operatori.
Introdurremo anche le \textbf{variabili}, che in C++ devono sempre avere un tipo, ed inizieremo ad illustrare la nozione di \textbf{visibilità} o \textbf{scope} delle variabili stesse.
Chiuderemo il capitolo illustrando l'importanza che i tipi hanno per l'individuazione degli errori che sono spesso presenti nei programmi che vengono scritti. 

\section{Tipi predefiniti}
\noindent I tipi predefiniti del C++ sono i seguenti: il tipo intero, due tipi per i reali, un tipo carattere, il tipo booleano ed il tipo \texttt{void}.
La Tabella 2.1 contiene alcune utili informazioni su questi tipi.
La colonna \texttt{BYTE} di questa tabella specifica quanti byte vengono usati dal compilatore GNU C++4.1 per rappresentare nella RAM i valori dei diversi tipi.
Questa quantità è importante perché essa fissa l'insieme dei valori supportati per ciascun tipo.
È importante chiarire subito che, benché, per esempio, l'insieme degli interi sia infinito in matematica, gli interi che un computer supporta saranno sempre in numero finito, visto che ogni computer ha una memoria finita.

\begin{itemize}
\item 
Per il tipo intero, l'intervallo dei valori rappresentabili con 4 byte (dal compilatore GNU 4.1) è, -2\ap{31} , ... , 2\ap{31} - l.
Le tipiche operazioni applicabili ai valori interi sono quelle aritmetiche ( + - / * e \% che rappresenta il modulo).
Oltre a queste sono applicabili ai valori interi anche le operazioni di confronto, cioè il test di uguaglianza ==, il test di disuguaglianza ! =, il test di maggiore >, il test di maggiore o uguale >=, eccetera.
L'ambiente C++ fornisce 2 costanti \texttt{INT\_MAX} e \texttt{INT\_MIN} che hanno come valore, rispettivamente, il massimo ed il minimo intero rappresentabile.
Useremo spesso queste costanti per inizializzare variabili intere con un valore che sia sicuramente il più grande o il più piccolo possibile.
Visto che gli interi rappresentati nel computer sono finiti, è naturale chiedersi cosa succede se sommando o moltiplicando 2 interi si esce dall'intervallo dei valori rappresentabili.
In questo caso diremo che si è verificato un errore di \textbf{overflow} ed il risultato dell'operazione è privo di senso.
Purtroppo l'overflow non viene, in generale, segnalato automaticamente dal computer.

\item
Per i reali il C++ offre 2 tipi: \texttt{double} e \texttt{float}.
Il tipo \texttt{double} che occupa 8 byte offre maggiore precisione del \texttt{float} per cui si usano solo 4 byte.
Una costante reale come \texttt{12.3} viene considerata dal compilatore di tipo \texttt{double}.
È necessario aggiungere \texttt{f} alla fine per renderla \texttt{float: 12.3f}.
La scrittura \texttt{2.3\ap{4}} indica il numero \texttt{double 2.3*10\ap{4}}.

\item
Le operazioni aritmetiche e quelle di confronto si possono applicare sia a valori \texttt{int}, che \texttt{float}, che \texttt{double}.
Chiariamo immediatamente che non si tratta delle stesse operazioni, ma solo di simboli uguali per indicare operazioni diverse.
La diversità è causata dalla diversa rappresentazione interna degli interi (complemento a 2) rispetto a quella dei reali (floating point con 4 o 8 byte).
Questo fenomeno di indicare con lo stesso simbolo operazioni simili concettualmente, ma che sono diverse a causa del tipo degli operandi, si chiama \textbf{sovraccaricamento} o \textbf{overloading}.
Vedremo nel seguito del capitolo come il compilatore decide qual'è l'operazione che deve venire effettivamente usata a fronte di un'operazione sovraccaricata nel programma che sta traducendo.

\item 
L'insieme di valori del tipo \texttt{char} è costituito da 256 caratteri.
Il compilatore C++ 4.1 della GNU usa un byte per rappresentare questi 256 valori ed infatti con otto bit si rappresentano 256 valori diversi che visti come interi in complemento a 2 sono gli interi da -128 a 127.
Quindi ogni carattere è rappresentato nella RAM da uno di questi interi.
I caratteri con codifica da O a 127 sono i caratteri ASCII e sono i caratteri alfabetici (minuscoli e maiuscoli), i numeri, i segni di punteggiatura, le parentesi e diversi caratteri che servono a controllare il cursore come il carattere di tabulazione e quello di invio.
I caratteri con codifica negativa dipendono invece dal sistema operativo che si utilizza.
In generale tra questi sono presenti molti caratteri accentati, il segno di insieme vuoto ed altri caratteri utili.
L'insieme dei 256 caratteri viene chiamato \textbf{extended ASCII}.
In \url{www.cplusplus.com} si possono trovare informazioni più precise. 

Visto che la rappresentazione interna dei caratteri è fatta attraverso interi(sebbene di 1 solo byte), ai caratteri si possono applicare le operazioni aritmetiche.
Naturalmente, in questo modo è molto facile ottenere risultati che non rappresentano più caratteri, cioè che sono fuori dall'intervallo -128... 27.
Come per gli errori di overflow è il programmatore che deve fare attenzione a questi possibili errori.

I valori di tipo carattere si rappresentano nei programmi tra apici, come per esempio in \texttt{'v'} , \texttt{'?'} e \texttt{'!'}, cf. la Tabella 2.1.
Ci sono caratteri che non sono visibili sullo schermo, ma hanno la funzione di spostare il cursore, per esempio, il carattere \texttt{'\textbackslash n'} serve per spostare il cursore all'inizio della prossima riga del video, mentre \texttt{'\textbackslash t'} muove il cursore al prossimo punto di tabulazione.
Con la crescente importanza di paesi che adottano alfabeti diversi dal nostro, sono nati tipi in grado di rappresentare insiemi di caratteri più ampi di 256.
A questo scopo il C++ prevede il tipo \texttt{wchar\_t} i cui valori occupano 2 o 4 byte. 

\item
Il tipo booleano, denotato \texttt{bool}, consiste solamente di 2 valori, \texttt{true} e \texttt{false}.
Ai valori di tipo \texttt{bool} si applicano gli operatori booleani che sono la congiunzione (AND), rappresentata con \texttt{\&\&}, la disgiunzione (OR), rappresentata con \texttt{||} e la negazione (NOT), rappresentata con \texttt{!}.
I valori booleani sono rappresentati internamente al computer con gli interi 0 (\texttt{false}) e 1 (\texttt{true}). 
Il fatto che i valori del tipo \texttt{bool} siano ripresentata con interi è un'eredità del C.
Visto che il C++ è compatibile col C, la stessa convenzione si applica anche al C++.
Come per i caratteri, il fatto che i valori booleani siano rappresentati internamente con interi, permette di applicare ai valori booleani tutte le operazioni applicabili agli interi.
Per esempio, \texttt{true - true == false} è un'espressione corretta del C++ ed ha valore \texttt{true} (con rappresentazione interna 1).
Visto che i valori booleani sono in realtà interi, anche la seguente espressione, \texttt{2 \&\& o \texttt{||} -4}, è un'espressione valida in C++ ed ha valore \texttt{true}.

\item
Il tipo \texttt{void} non ha valori né operazioni.
Esso viene usato proprio per indicare la mancanza di valori.
Vedremo il suo uso nelle funzioni che non restituiscono alcun risultato, cf. Sezione 7.3.

\end{itemize}

In generale, per sapere quanti byte occupano i valori di un qualsiasi tipo \texttt{T} possiamo utilizzare la funzione di libreria \texttt{sizeof}.
L'istruzione, \texttt{cout $<<$ sizeof(T);} stampa sul video il numero di byte usati per un qualsiasi valore del tipo \texttt{T}.
Oltre ai tipi predefiniti elencati nella Tabella 2.1, il C++ permette anche i tipi \texttt{short int, long int} e \texttt{long double}.
Il significato dei qualificatori \texttt{short} e \texttt{long} è ovvio, ma in molte realizzazioni i valori di questi tipi occupano lo stesso numero di byte del tipo base a cui il qualificatore è applicato.
Per esserne certi si può usare la funzione \texttt{sizeof()} descritta prima. 

Un altro tipo di valore, usato molto frequentemente nei programmi, è la stringa (di caratteri). Un tale valore è usato nell'Esempio 1.1 che scrive sul video la stringa, \texttt{''inserire 2 interi''}.
Queste stringhe si chiamano \textbf{stringhe alla C}, per contrasto rispetto a quelle \textbf{alla C++} che sono realizzate dal tipo contenitore string di cui parleremo nella Sezione 9.6.
Le stringhe alla C sono dei valori costanti formati da sequenze di caratteri racchiuse tra doppi apici.
Esse possono contenere qualunque carattere, anche quelli di controllo.
Per esempio, se nell'Esempio 1.1 sostituissimo nel la seconda riga del ma in la seguente stringa: \texttt{''inserire 2 interi \textbackslash n''} a quella originale, potremmo osservare che l'output prodotto dal nuovo programma cambierebbe rispetto a quello del programma originale.
Le stringhe alla C sono menzionate anche nella Sezione 5.1. 

\section{Variabili e dichiarazioni}
\noindent Come tutti i linguaggi di programmazione, il C++ permette di usare dei nomi per rappresentare i dati che i programmi manipolano.
Questi nomi si chiamano \textbf{variabili}(o anche identificatori).
Le variabili devono iniziare sempre con un carattere alfabetico (minuscolo o maiuscolo) oppure con il carattere di sottolineatura \_ (un­derscore) ed i caratteri successivi possono essere o alfabetici o numerici oppure il carattere \_.
La massima lunghezza delle variabili non è fissata, ma nomi troppo lunghi possono venire abbreviati in fase di compilazione e comunque rischiano di appesantire la programmazione.
Il C++ distingue i caratteri minuscoli da quelli maiuscoli.
Quindi \texttt{pippo} è una variabile diversa da \texttt{Pippo} ed entrambe sono diverse da \texttt{PIPPO}. 

Le variabili vengono introdotte nei programmi C++ attraverso le operazioni di \textbf{dichiarazione} che specificano il tipo delle variabili dichiarate.
Per esempio, la dichiarazione \texttt{int x, pippo;} specifica che \texttt{x} e \texttt{pippo} sono variabili di tipo \texttt{int}(o semplicemente sono variabili intere).
In generale, se il programma è corretto, una tale dichiarazione garantisce che i valori che, durante l'esecuzione del programma, verranno associati alle due variabili \texttt{x} e \texttt{pippo} saranno di tipo \texttt{int}.
Al momento della dichiarazione di una variabile, è anche possibile (non necessario) assegnare un valore alla variabile stessa, come per esempio in: \texttt{int x=O, pippo= -1;}.
In questo caso parliamo di \textbf{inizializzazione} delle variabili.
Una variabile dichiarata, ma non inizializzata, è \textbf{indefinita}.
Esempi di dichiarazioni di variabili di tipo char sono: \texttt{char y1, z;} e \texttt{char y1=' a' , z;} in cui \texttt{y1} viene inizializzata al carattere \texttt{'a'}.
I nomi delle variabili devono in ogni caso essere diversi dalle parole chiave (keyword) del C++ come, per esempio, \texttt{int} e \texttt{char} e altre che saranno introdotte nel seguito. 

A volte nei programmi c'è l'esigenza di usare dei valori \textbf{costanti}, per esempio costanti numeriche, stringhe particolari, eccetera.
Piuttosto che usare direttamente i valori costanti nel testo del programma conviene dichiarare nomi costanti a cui assegnare questi valori e poi usare i nomi nel testo del programma dovunque servano i valori.
Le dichiarazioni di costante hanno la seguente forma: \texttt{const double pi=3.14;} oppure \texttt{const char inizio='a', fine='z';}.
Quindi si tratta semplicemente di premettere la parola chiave \texttt{const} ad una normale dichiarazione.
Importante notare che nel caso delle costanti l'inizializzazione è sempre richiesta al momento della dichiarazione.
Il valore assegnato(ovviamente) non potrà venire modificato nel programma (pena un messaggio d'errore del compilatore).
Usare dichiarazioni di costanti, rispetto ad usare direttamente i valori costanti nel testo del programma, fa guadagnare in leggibilità, in robustezza ed in facilità di modifica dei programmi. Il guadagno in leggibilità è ovvio se i nomi costanti sono scelti con intelligenza.
La robustezza deriva dal fatto che accidentali modifiche di un valore che dovrebbe essere costante vengono segnalate.
Per la modificabilità, è ovvio che, in caso un valore costante sia da cambiare, è certamente più semplice e sicuro modificare la corrispondente dichiarazione di costante piuttosto che modificare tutte le occorrenze del valore nel testo del programma. 

\section{Espressioni e conversioni}
Un'espressione è composta nel caso più semplice da costanti, per esempio, \texttt{2*( 5+24)}, il cui valore è 58.
Espressioni possono contenere variabili, per esempio, \texttt{x*(pippo+24)}, il valore di questa espressione lo si calcola sostituendo ad ogni variabile il suo R-valore.
Supponiamo che x abbia R-valore 5 e \texttt{pippo} abbia R-valore 2, in questo caso l'espressione ha valore 130.
Sottolineiamo un fatto semplice, ma importante: le espressioni nei programmi vengono sempre valutate e questo procedimento (se ha successo) produce il \textbf{valore} dell'espressione.
E' possibile che la valutazione di un'espressione non abbia successo?
La risposta è affermativa e il fallimento della valutazione può avvenire per vari motivi.
La ragione più semplice è che l'espressione richieda di eseguire operazioni indefinite come la divisione per O.
Questo errore produce generalmente una terminazione anormale del programma in esecuzione.
Un altro possibile motivo che può impedire la valutazione di un'espressione è che l'espressione contenga operandi ed operatori di tipi incompatibili.
Una tale espressione è \texttt{"pippo" * 13} in cui si chiede di moltiplicare una stringa alla C (il cui tipo non conosciamo ancora) con un intero.
Questa espressione ovviamente non ha senso e questo fatto ci verrà segnalato dal compilatore.
Quindi un programma che contiene una take espressione non viene (in generale) compilato e non potrà quindi venir eseguito.
Un caso ancora diverso è quello in cui l'espressione da valutare contiene variabili indefinite.
In questa situazione la valutazione dell'espressione può venir eseguita senza errori apparenti ed il solo sintomo dell'errore è un valore finale casuale.

Nel seguito, considereremo cosa avviene nel caso di espressioni che mescolano valori e variabili di tipi diversi, ma non incompatibili tra loro, cioè espressioni che il compilatore riesce a tradurre in codice oggetto senza dare errori.
Questa situazione è problematica perché le operazioni macchina si applicano su valori dello stesso tipo.
Per esempio, i computer possono eseguire l'operazione di somma di 2 operandi interi e anche la somma di 2 operandi reali, ma non la somma tra un intero ed un reale.
Quindi cosa fa il compilatore se deve tradurre un'espressione come \texttt{2 + 3.14}?
Ovviamente ci sono 2 possibilità: o trasformare l'intero 2 in un reale (scrivendo 2 in forma floating point) oppure trasformare il reale 3.14 in un intero, per esempio troncando la parte decimale riducendo 3.14 a 3.
Il compilatore C++ sceglie sempre la prima possibilità.
Il motivo è semplice e logico: un intero occupa meno byte (4) di un \texttt{double} (8 byte) e quindi la conversione dell'intero in \texttt{double} non comporta mai una perdita di informazione: ogni intero tra -2\ap{31} e 2\ap{31}-1 ha una rappresentazione precisa in floating point con 8 byte.
E' chiaro che invece trasformare 3.14 in un intero ci faccia perdere i decimali .14.

Quindi il compilatore C++ quando compila espressioni con valori di tipi diversi e compatibili, trasforma alcuni di questi valori in modo che ogni operazione si applichi a valori dello stesso tipo e nel farlo segue il seguente semplice principio: \textbf{vengono applicate le conversioni che producono il minimo rischio di perdita di informazione}.
Le trasformazioni di tipo che soddisfano questo principio, vengono chiamate \textbf{promozioni}.
In Sezione 9.4 si possono trovare maggiori dettagli sulle promozioni e in generale su tutti i tipi di conversione e su cosa comporti in pratica convertire un valore da un tipo ad un altro tipo. 

Espressioni molto usate nei programmi sono le espressioni booleane, cioè quelle il cui valore è di tipo booleano.
Queste espressioni generalmente usano gli operatori relazionali come \texttt{$> \; < \; >= \; <= \; ==$} (uguale) \texttt{!$=$} (diverso) e gli operatori logici \texttt{\&\& || !}.
Un esempio di espressione booleana è la seguente.
Si assuma che \texttt{x} sia una variabile di tipo char: \texttt{(x>='a') \&\& (x<='z')}, ha valore \texttt{true} se \texttt{x} ha come R-valore un carattere alfabetico (tra 'a' e 'z') ed altrimenti l'espressione ha valore false.
Si noti che questa espressione usa il fatto che la codifica ASCII dei caratteri assegna ai caratteri alfabetici dei valori interi contigui e coerenti con l'ordine alfabetico.
Più precisamente, 'a' ha codice ASCII 97, 'b' 98 e così via.
Naturalmente il codice ASCII codifica in modo simile anche i 10 caratteri numerici e le maiuscole.

Per le espressioni booleane che contengono gli operatori logici \texttt{\&\&} e \texttt{$||$}, il C++ usa la cosiddetta valutazione \textbf{abbreviata} (o \textbf{shortcut}).
Consideriamo l'espressione esaminata prima \texttt{(x>='a') \&\& (x<='z')}.
La valutazione procede da sinistra a destra, ma se la condizione di sinistra \texttt{x>='a'} è falsa allora non viene valutata la condizione di destra.
Infatti l'intera espressione è falsa, indipendentemente dal valore della condizione di destra.
Nel caso di un'espressione che contiene l'operatore \texttt{||}, come per esempio, \texttt{(x<'a') || (x>'z')}, se la condizione di sinistra è vera non viene valutata quella di destra perché l'intera espressione è vera indipendentemente dal valore della condizione di destra.

\section{A cosa servono i tipi}
In C++, così come in moltissimi linguaggi di programmazione, ogni variabile deve venire dichiarata prima di poter essere usata e la dichiarazione specifica il tipo della variabile.
Questa associazione implica (se il programma è corretto rispetto ai tipi) che, durante ogni esecuzione del programma, ogni variabile assumerà solo R-valori appartenenti al tipo dichiarato per quella variabile.
Un tale comportamento è una restrizione che vincola la libertà del programmatore, ma è una restrizione \textit{a fin di bene} per chi programma.
Infatti, grazie al tipo di ciascuna variabile, diventa più facile evitare di inserire errori nei propri programmi (per esempio, scrivendo espressioni prive di senso che mescolano variabili e valori di tipi incompatibili).
Un vantaggio ancora più importante è che il compilatore usa i tipi per controllare automaticamente che i nostri programmi usino le variabili solo in modo coerente rispetto al tipo dichiarato per esse.
Inoltre i tipi servono al compilatore per risolvere i problemi di overloading.
Si immagini che il compilatore debba tradurre l'espressione \texttt{x + y}.
Quale operazione di somma dovrà inserire nel codice macchina che deve produrre?
Quella che somma 2 interi o quella che somma 2 reali?
I tipi di \texttt{x} e \texttt{y} dicono al compilatore cosa deve fare.
Il caso di tipi diversi tra loro è discusso in Sezione 2.3.

Nonostante la parte orientata agli oggetti del C++ non venga trattata in questo testo, quando si discute dell'utilità dei tipi è obbligatorio accennarvi.
Gli oggetti sono istanze di tipi (chiamati classi) che racchiudono al loro interno dati assieme a funzioni per manipolarli.
Lo scopo è quello di proteggere i dati e di non permettere che essi vengano manipolati in modo arbitrario con funzioni diverse da quelle previste a questo scopo nella classe.
La nozione di classe e la relazione di ereditarietà tra classi sono un importante ausilio alla realizzazione di programmi corretti e modificabili.

Occorre osservare però che il C++ (soprattutto a causa dell'eredità del C) non è un linguaggio \textit{virtuoso} nella gestione dei tipi.
Un anticipo di questo approccio troppo liberale rispetto ai tipi, l'abbiamo già avuto per i tipi predefiniti.
Per esempio, in C++ i valori di tipo carattere e di tipo booleano vengono rappresentati internamente come interi e questo fatto rende corretto applicare a valori di tipo carattere e booleano degli operatori che una disciplina più rigorosa sui tipi considererebbe errori.
Quindi, abbattendo le differenze tra tipi diversi, si perde la capacità di scoprire errori grazie ai tipi.
Oltre a quelle appena segnalate, nel C++ ci sono altre debolezze, ancora più gravi, che vedremo successivamente e che rendono il C++ un linguaggio \textbf{insicuro rispetto ai tipi}, cioè un linguaggio nel quale è possibile (e purtroppo anche molto facile) scrivere programmi che (durante l'esecuzione assegnano ad una variabile di un certo tipo degli R-valori di tipo diverso (e incompatibile) senza che queste violazioni vengano segnalate né durante la compilazione né durante l'esecuzione.