\chapter{Puntatori, array ed aritmetica dei puntatori}

I puntatori sono una nozione fondamentale per il C. Nel C++ essi sono affiancati dai riferimenti. In questo capitolo illustreremo entrambe le nozioni sottolineando la stretta relazione esistente tra di esse e spiegando il motivo per cui i riferimenti hanno soppiantato i puntatori nei linguaggi più moderni come Java. Successivamente introdurremo gli array assieme al costrutto iterativo \textsc{for} che si usa spesso per percorrere gli array. Cura particolare sarà dedicata a spiegare il tipo degli array a più dimensioni. Aver chiarito questo argomento tornerà molto utile per capire \textit{l'aritmetica dei puntatori} che è descritta nella sezione finale del capitolo.

\section{Puntatori e riferimenti}

\section{Array e for}

\section{Il tipo degli array}

\section{L'aritmetica dei puntatori}

\section{Stringhe alla C e array di caratteri}

\newpage