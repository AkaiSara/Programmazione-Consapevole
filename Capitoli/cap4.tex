\chapter{La correttezza dei programmi}
\noindent In questo capitolo mostreremo come sia possibile dimostrare in modo convincente che i nostri programmi sono corretti, cioè che fanno effettivamente quello per cui li abbiamo scritti.
Inizieremo dall'Esercizio 3.3 che ci aiuterà a definire una regola generale per dimostrare la correttezza dei cicli while e do-while.
Con questa regola affronteremo l'Esercizio 4.2 che già presenta maggiori difficoltà. 

Molto spesso il programmatore realizza il suo programma seguendo un'idea intuitiva più o meno precisa.
Successivamente controlla il programma eseguendo alcuni test con input diversi.
Questa maniera di procedere, molto comune, non garantisce che il programma sia corretto.
Sviluppare una prova di correttezza consiste nel precisare l'idea intuitiva, evidenziando eventuali errori.
Nel seguito della trattazione cercheremo di sottolineare i passaggi in cui la sua formalizzazione rivela gli errori nascosti sotto un'intuizione troppo superficiale.

\bigskip
\bigskip

\paragraph{Esercizio Risolto 4.1}
\textit{Nell'Esercizio Risolto 3.3 abbiamo presentato 2 soluzioni per risolvere uno stesso esercizio
Vogliamo ora dimostrare che sono entrambe corrette.
Per dimostrare che un programma è corretto, è necessario prima scrivere in maniera precisa cosa il programma dovrebbe fare.
Si tratta di specificare due asserzioni, la pre- e la postcondizione del programma.
La precondizione specifica la situazione dello stato del calcolo che si assume valga quando il programma inizia la sua esecuzione.
La postcondizione invece specifica la situazione dello stato del calcolo dopo che il programma termina.
La dimostrazione di correttezza di un programma è sempre in relazione ad una coppia di pre- e postcondizioni e consiste nel dimostrare che se il programma esegue partendo da uno stato del calcolo che soddisfa la precondizione, allora, se esso terminerà, lo farà in uno stato che soddisfa la postcondizione.
Vediamo di specificare pre- e postcondizioni per il problema in esame:}

\begin{itemize}

\item
\textbf{PRE = (\texttt{cin} contiene l'intero 0 preceduto da \texttt{n} interi diversi da 0, con \texttt{n$\geq$}0)}

\item
\textbf{POST = (\texttt{neg+post=n} e \texttt{neg} è il numero degli interi negativi che precedono 0 su \texttt{cin}, mentre \texttt{pos} è il numero degli interi positivi che precedono 0)}

\end{itemize}

\noindent \textit{La precondizione asserisce che su \texttt{cin} c'è certamente uno 0 e che esso è preceduto da zero o più interi diversi da 0.
Questo è importante perché ci assicura che il programma terminerà sempre.
La postcondizione invece asserisce che il programma deve calcolare in \texttt{neg} il numero dei negativi e in \texttt{pos} quello dei positivi che precedono lo 0.
Ovviamente i nomi \texttt{pos} e \texttt{neg} sono arbitrari.
Per dimostrare che la prima soluzione dell'Esercizio Risolto 3.3), quella che usa il \texttt{while}, è corretta, dobbiamo dimostrare che il suo ciclo \texttt{while}, iterazione dopo iterazione arriva ad attribuire a \texttt{neg} e \texttt{pos} il valore giusto rispetto ai valori letti da \texttt{cin}.
Possiamo dire che ogni volta che stiamo per ripetere il ciclo \texttt{while}, avremo letto \texttt{neg+pos+1} interi da \texttt{cin} e i primi \texttt{neg+pos} sono diversi da 0 e ce ne sono \texttt{neg} negativi e \texttt{pos} positivi, e inoltre, se \texttt{x} è 0 allora \texttt{neg+pos=n}, mentre se è diverso da 0, allora \texttt{neg+pos$<$n}.}
\[
\text{R = }
\begin{cases}
(1) & \textbf{(letti \texttt{neg+pos+1} valori da \texttt{cin} (neg e pos non negativi))} \\
(2) & \textbf{(i primi \texttt{neg+pos} valori letti sono tutti diversi da 0} \\
	& \textbf{ e ce ne sono \texttt{neg} negativi e \texttt{pos} positivi)} \\
(3) & \textbf{x=0 $\Rightarrow$ \texttt{neg+pos=n})} \\
(4) & \textbf{x $\neq$ $\Rightarrow$ \texttt{neg+pos<n})}
\end{cases}
\]
\textit{Osserviamo alcune cose interessanti su questa} \textbf{R}:

\begin{enumerate}


\item
\textit{la prima volta che l'esecuzione arriva al ciclo (cioè prima di valutare il test di permanenza del ciclo), \textbf{R} vale:
abbiamo letto un solo valore e infatti \texttt{neg+pos=}0 (vale(1) di \textbf{R}) e se \texttt{x=}0, allora \textit{n} è 0 e quindi \texttt{neg+pos=}0 (e quindi vale (3) di \textbf{R}), mentre se \texttt{x$\neq$}0, allora \texttt{n>}0 (e quindi vale (4) di \textbf{R}).
Il punto (2) di \textbf{R} vale in modo banale.}

\item
\textit{Se \textbf{R} vale all'inizio del ciclo e il test di permanenza è vero allora dopo aver eseguito il corpo del ciclo una volta di più, vale ancora \textbf{R}.
Vista l'ipotesi che il tesi di permanenza è vero, \texttt{x} non è 0 e quindi è un valore che precede 0 per cui il test aumenta correttamente o \texttt{neg} o \texttt{pos}) ((2) di \textbf{R}).
Poi viene letto un nuovo valore, ma visto che \texttt{neg+pos} è aumentata di 1, \texttt{neg+pos+1} è ancora uguale al numero di letture (punto (1) di \textbf{R}).
(3) e (4) di \textbf{R} valgono banalmente.
Questo significa che \textbf{R} vale ogni volta che l'esecuzione arriva all'inizio del ciclo.
Per questo motivo \textbf{R} è detto} \textbf{invariante} \textit{del ciclo}.

\item
\textit{Se all'inizio del ciclo vale \textbf{R} e il test di permanenza è falso, allora vale \textbf{POST}.
Che il test di permanenza sia falso significa che \texttt{x=}0 e in questo caso i punti (2) e (3) di \textbf{R} implicano \textbf{POST}.}

\end{enumerate}

\textit{I tre punti appena visti formano lo schema che useremo sempre per dimostrare la correttezza dei cicli.
Chiaramente il cuore della dimostrazione è l'invariante \textbf{R} che descrive quello che il ciclo calcola dopo ogni numero di iterazioni.
Ma possiamo seguire questo schema anche per dimostrare la correttezza della seconda soluzione 
dell'Esercizio Risolto 3.3, quella che usa il \texttt{do-while}?
\textbf{PRE} e \textbf{POST}}



\section{Regola di prova del \texttt{while}}

\section{Altri esempi di prove di correttezza}

\subsection{Esempio di gestione delle eccezioni}

