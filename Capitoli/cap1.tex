\chapter{Introduzione}
\noindent
Questo libro è rivolto a studenti che sanno poco o nulla di programmazione e che intendono impararla in modo approfondito.
Il suo scopo non è solamente quello di illustrare un particolare linguaggio di programmazione attraverso una serie di esempi, ma anche quello di proporre un modo \textit{consapevole} di scrivere programmi che aumenti la probabilità che siano subito corretti.
Pensiamo che un buon informatico debba sapere cosa succede quando i suoi programmi sono compilati ed eseguiti e quindi il libro tratta anche alcuni concetti relativi ai linguaggi di programmazione che raramente sono trattati in un testo introduttivo.
Tra questi concetti ricordiamo i tipi, la gestione dei dati durante l'esecuzione dei programmi, la gestione delle eccezioni, e la gestione dell'esecuzione delle funzioni (anche ricorsive).
L'importanza di questi concetti va ben al di là del singolo linguaggio (che sia C o C++), infatti essi si applicano praticamente a tutti i linguaggi di programmazione esistenti. 

Per quanto riguarda il modo di sviluppare programmi, l'idea che permea questo testo è che più un programma è semplice, più è probabile che non contenga errori.
Non a caso il sottotitolo del libro è \textit{semplice è bello}.
La convinzione dell'autore (e che pare difficile confutare) è che:
\textit{non si può scrivere un programma e credere che sia corretto se non si sa dire chiaramente cosa il programma dovrebbe fare e perché lo fa}.
In coerenza con questo principio, ogni programma che realizzeremo sarà accompagnato da una descrizione precisa di quello che vorremmo che calcolasse e da una dimostrazione che effettivamente lo calcola.
Queste dimostrazioni sono comprensibili e quindi utili solo quando, nello scrivere i programmi, avremo seguito una logica il più possibile semplice e lineare.
Insomma, il motto \textit{semplice è bello} è una scelta obbligata se vogliamo dimostrare la correttezza dei nostri programmi.

Il linguaggio di programmazione adottato nel testo è il C++ senza le classi e gli oggetti.
Non si tratta del C, infatti anche senza le classi e gli oggetti, il C++ ha alcune caratteristiche che lo rendono più espressivo del suo predecessore C, come per esempio il passaggio dei parametri per riferimento, e la gestione delle eccezioni.
Per non ripetere continuamente il\textit{C++ senza le classi e gli oggetti}, nel seguito chiameremo questo linguaggio semplicemente C++.
Per quanto riguarda caratteristiche del C++ che non hanno trovato posto in questo libro, esistono molti manuali di riferimento del linguaggio, alcuni dei quali sono disponibili gratuitamente su internet.
Anche il sito (\url{www.cplusplus.com}) è ricco di informazioni.

Il libro è organizzato in 12 Capitoli.
I Capitoli 2, 3, 5, 7 e 9 trattano il linguaggio di programmazione.
Ovviamente nei primi Capitoli sono illustrate le cose più semplici come i tipi predefiniti e le istruzioni di base, per passare successivamente ai tipi definiti dall'utente e ai contenitori, per finire con il sovraccaricamento e i \texttt{namespace}.
Questi Capitoli sono intercalati da Capitoli con molti esempi ognuno corredato dalla relativa dimostrazione di correttezza.
Il Capitolo 4 introduce le prime dimostrazioni di correttezza, ma il Capitolo 6 riveste un ruolo particolarmente importante in quanto in esso viene formulata una semplice \textit{ricetta}, detta \textit{ricetta di} \textbf{indicizzazione}, che è molto utile in tutte le dimostrazioni successive.
La ricorsione è introdotta nel Capitolo 10 e approfondita nei successivi Capitoli 11 e 12 che illustrano le liste concatenate e gli alberi binari.
Questi Capitoli contengono molti esempi di funzioni ricorsive la cui correttezza è dimostrata con l'induzione. 

In quello che resta del presente Capitolo verranno introdotte alcune nozioni molto semplici, ma comunque molto utili nel seguito del testo, su come è organizzato un Computer, su come scrivere un programma C++ e su come eseguirlo. 
Il Capitolo si conclude con la Sezione 1.3 in cui viene spiegata la notazione logico-matematica che è usata costantemente nel resto del libro. 

\section{Struttura di un computer}
\noindent
L'architettura di un computer moderno è sorprendentemente simile a quella dei primi computer costruiti negli anni quaranta.
Questa architettura viene chiamata di von Neumann in onore dell'ideatore dei primi computer. Naturalmente la tecnologia di realizzazione dei computer si è enormemente evoluta nel corso di questi 70 anni, e questa evoluzione ha portato un'incredibile riduzione nelle dimensioni dei computer ed un altrettanto incredibile aumento della loro velocità di esecuzione, ma le funzionalità delle diverse parti e la loro cooperazione sono, in linea di principio, ancora simili a quelle dei tempi di von Neumann. 

Un computer contiene un'unità di calcolo ed una memoria.
L'unità di calcolo è detta \textbf{Central Processing Unit} (CPU) e consiste di registri capaci di contenere i valori da manipolare durante l'elaborazione e da circuiti in grado di effettuare operazioni (per esempio, la somma o la sottrazione) sui valori contenuti nei registri e anche di trasferire valori dalla memoria ai registri e viceversa.
Le operazioni che questi circuiti sono in grado di effettuare sono dette \textbf{operazioni macchina}.
La memoria è costituita da una sequenza di byte, ciascuno composto da 8 bit e identificato da un indirizzo.
Gli indirizzi dei byte della memoria iniziano da 0 e crescono fino alla dimensione totale della memoria stessa.
Un valore, per esempio un valore intero, può occupare più byte contigui e in questo caso diremo che il valore risiede nella memoria all'indirizzo del primo di questi byte.
Abbiamo già detto che la CPU è in grado di eseguire operazioni che trasportano valori dalla memoria RAM ad un registro della CPU e viceversa.
Queste operazioni richiedono di specificare l'indirizzo del primo byte da cui si preleva (o in cui viene copiato) il valore trasportato.
Dato che tramite gli indirizzi è possibile accedere ad un qualsiasi byte della memoria, questa memoria è chiamata \textbf{Random Acccess Memory (RAM)}, cioè Memoria ad Accesso Casuale.
Per quanto riguarda l'esecuzione dei programmi da parte di un computer, essa avviene in linea di massima nel modo seguente:
il programma da eseguire ed i valori che esso manipola durante l'esecuzione, si trovano entrambi nella RAM (ma in parti diverse).
Uno dei registri della CPU, detto \textbf{Program Counter} (PC) contiene in ogni momento l'indirizzo della prossima istruzione del programma da eseguire e la CPU \textit{riconosce} quell'istruzione, la esegue e passa alla istruzione successiva incrementando il PC.
L'esecuzione di un'istruzione in generale modifica i valori dei dati del programma.

Ovviamente questa descrizione è molto semplificata, ma sufficiente per capire i concetti relativi alla programmazione che seguiranno.

\section{Come scrivere ed eseguire un programma}
\noindent 
Ogni programma, scritto in un qualsiasi linguaggio di programmazione (come Pa­scal o C++ o Java, ecc.), per poter essere eseguito deve venire tradotto in una sequenza di operazioni macchina.
Questa traduzione viene effettuata da un programma che si chiama \textbf{compilatore}.
Il programma di partenza (in Pascal o C++ ecc.) viene chiamato \textbf{programma sorgente}, mentre la corrispondente traduzione in linguaggio macchina si chiama \textbf{programma oggetto}\footnote[1]{In realtà la traduzione dal linguaggio sorgente a quello macchina può consistere di vari passaggi che coinvolgono linguaggi intermedi, macchine virtuali e fasi di interpretazione, ma questi aspetti esulano dallo scopo del presente testo.}.

La traduzione del programma sorgente in programma oggetto ha successo solo se il sorgente è scritto seguendo alla lettera le regole sintattiche del linguaggio di programmazione adottato.
Queste regole sono molto precise in quanto esse devono eliminare qualsiasi ambiguità nel significato dei programmi.
Per esempio nel C++ ogni istruzione deve terminare con un ';' ed i blocchi con più di una istruzione devono essere racchiusi tra parentesi graffe.
Violazioni alle regole sintattiche vengono segnalate dal compilatore con messaggi che generalmente consentono di correggere facilmente gli errori. 

Per eseguire i programmi presentati in questo libro, basterà avere sul proprio computer un compilatore C++ possibilmente conforme al C++ standard.
Consigliamo il compilatore GNU C++ 4.1 (e qualunque versione successiva).
Si tratta di software libero che è facile trovare e scaricare dalla rete sul proprio computer (\url{gcc.gnu.org}).
Questo compilatore è realizzato per funzionare con il sistema operativo Linux, ma per i sistemi operativi Microsoft è possibile scaricare l'ambiente \textbf{Cygwin} che simula un ambiente Linux e possiede al suo interno un compilatore C++ della GNU.
Anche Cygwin è software libero (\url{Cygwin.com}).
Nel seguito assumeremo che i nostri lettori siano in grado di aprire una finestra di comando (detta \textbf{shell}), da cui potranno spostarsi nelle diverse cartelle del file system che contengono i loro programmi e da cui potranno lanciare i comandi di compilazione e di esecuzione dei programmi, che descriveremo dopo il seguente esempio. 

\bigskip

\noindent\textbf{Esempio 1.1}
\begin{codice}

#include<iostream>
using namespace std;
main()
{
  int x, y;
  cout << "Inserire 2 interi";
  cin >> x >> y;
  cout << "Valore di x = " << x << ", valore di y = " << y;
}
\end{codice}

\paragraph{Esercizio 1.2} \textit{Scrivere il programma dell'Esempio 1.1 usando un qualsiasi editore di lesto (si consiglia} \texttt{xemacs})\textit{, compilarlo ed eseguirlo utilizzando i comandi spiegati di seguilo.}

Assumiamo che il file di testo contenente il programma dell'Esempio 1.1 si chiami \texttt{pippo.cpp} e che risieda nella cartella \texttt{pluto}.
Useremo spesso nomi come \texttt{pippo} e \texttt{pluto} per indicare che si tratta di nomi arbitrari.
Al contrario, l'estensione \texttt{.cpp} in \texttt{pippo.cpp} è necessaria perché indica al compilatore che il file contiene un programma C++.
Se la cartella corrente è \texttt{pluto}, il comando \texttt{g++} \texttt{pippo.cpp}, invoca il compilatore C++ che compila il programma \texttt{pippo.cpp} e produce il corrispondente programma oggetto nel file \texttt{a.out} o \texttt{a.exe}, a seconda che il nostro computer abbia un sistema operativo Linux oppure Windows, rispettivamente.
Una volta effettuata la compilazione, il comando \texttt{./ a.out} (oppure \texttt{./ a.exe}) esegue il codice oggetto.
Qualora si desideri attribuire un nome meno anonimo di \texttt{a.out} o \texttt{a.exe} al file oggetto, basterà lanciare la compilazione con il comando seguente:
\texttt{g++ -o paperino pippo.cpp}.
Questo comando metterà il codice oggetto nel file \texttt{paperino}, che verrà eseguito con \texttt{./ paperino}.
Qualunque sia il nome del codice oggetto, la sua esecuzione causerà l'apparizione sul video della scritta, \texttt{Inserire 2 interi}, con il cursore immediatamente dopo che lampeggia in attesa dei 2 interi richiesti.
Una volta inserita da tastiera una coppia di interi, per esempio \texttt{3} e \texttt{-2} (separati da uno spazio o da un invio e seguiti da un invio), sul video apparirà la scritta \texttt{Valore di x = 3} seguita da \texttt{, valore di y = -2}.
Il programma è tale che l'input e l'output, entrambi sul video, siano accostati in modo poco curato.
Negli esempi successivi vedremo alcuni semplici accorgimenti per realizzare output di qualità migliore.
Nonostante la sua semplicità, ci sono parecchie cose da spiegare nel programma dell'Esempio 1.1: 

\begin{enumerate}
\item 
La prima istruzione \texttt{\#include<iostream>} non è un'istruzione C++ bensì un comando, rivolto al compilatore, che richiede l'inclusione della classe di input/output \texttt{iostream} nel \texttt{namespace std}.
La successiva istruzione \texttt{using namespace std;} permette al nostro programma di accedere a tutto quello che è definito nel \texttt{namespace std} e quindi anche alla classe \texttt{iostream}.
Usare questa classe è necessario perché essa contiene le definizioni delle operazioni di input e di output che usiamo nel nostro programma e quindi se la classe \texttt{iostream} non fosse presente le operazioni di input e di output risulterebbero indefinite.
Tutto questo sembrerà certamente \textit{ostrogoto} al neofita.
Per il momento si tratta di usare queste istruzioni senza capirle.
La successiva trattazione dovrebbe renderle più comprensibili. 

\item
\texttt{main()} è il nome di una \textbf{funzione} e quello che segue, racchiuso tra parentesi graffe, è il \textbf{corpo} della funzione, cioè le istruzioni che la compongono.
Nel seguito una sequenza di istruzioni racchiuse tra graffe viene chiamato un \textbf{blocco}.
In inglese \texttt{main} significa \textit{principale}.
Ogni programma deve contenere esattamente una funzione \texttt{main} e l'esecuzione di ogni programma inizia sempre dalla prima istruzione della funzione \texttt{main}.
Le parentesi tonde \texttt{()} che seguono il nome \texttt{main} indicano che si tratta di una funzione e che (in questo caso) non ha argomenti.
In realtà il C++ consente di passare parametri al \texttt{main}, ma è decisamente troppo presto per questo genere di dettagli tecnici.
Rinviamo il lettore inguaribilmente curioso al sito \url{www.cplusplus.com} per maggiori informazioni su questo.

\item
Nel C++ l'input e l'output sono visti come sequenze di byte (come la memoria RAM).
In inglese sequenza si dice "stream" ed infatti i nomi legati all'input ed all'output sono variabili di tipo \texttt{stream}.
L'operazione di out­put più comune è $<<$.
Il dispositivo standard di output è il video ed il nome dello \texttt{stream} che gli viene associato nella libreria \texttt{iostream} è \texttt{cout}.
Per l'input, l'operazione base è $>>$, il dispositivo standard di input è la tastiera ed il nome dello \texttt{stream} associato è \texttt{cin}.

\item
Nel programma che stiamo considerando usiamo sia l'operazione di scrittura $<<$ che quella di lettura $>>$.
Queste operazioni hanno entrambe 2 argomenti:
lo \texttt{stream cout} o \texttt{cin} ed il valore da mandare in output oppure, rispettivamente, la variabile a cui assegnare il valore letto da \texttt{cin}.
Un esempio di un comando di output è la seconda riga del corpo del \texttt{main}:
questa istruzione (quando viene eseguita) produce la stampa su video del valore stringa di caratteri \texttt{"Inserire 2 interi"}.
Nella terza riga del \texttt{main} troviamo invece un esempio di un'istruzione di lettura.
Questa istruzione va vista in realtà come \texttt{(cin >> x) >> y;} in cui la prima parte \texttt{(cin >> x)} legge il prossimo valore dallo \texttt{stream} di input \texttt{cin}, lo assegna alla variabile \texttt{x} e restituisce \texttt{cin} in modo da consentire la seconda lettura, \texttt{(cin >> y)}, che assegna il prossimo valore letto a \texttt{y} e restituisce \texttt{cin} che però non serve a nulla dato che le letture sono finite.
Per finire, la quarta istruzione del corpo del \texttt{main} è un output che stampa di seguito le stringhe \texttt{"Valore di x = "} e \texttt{", valore di y = "}, ciascuna seguita dal corrispondente valore (appena letto) di \texttt{x} e di \texttt{y}. 

\end{enumerate}

\section{Notazione}
\noindent Per descrivere in modo formale i concetti ed in particolare per scrivere le asserzioni di correttezza dei programmi, useremo una notazione logico-matematica che conviene spiegare una volta per tutte in questa Sezione.

Principalmente useremo le operazioni logiche di congiunzione (e), di disgiunzione (o), di negazione, d'implicazione e di equivalenza.
Per completezza, ricordiamo che la congiunzione di 2 affermazioni (A e B) è vera quando entrambe sono vere, mentre la loro disgiunzione (A o B) è vera se almeno una tra A e B è vera.
La negazione di un'affermazione è vera se l'affermazione è falsa.
Un'affermazione A implica B, quando B è vera ogni volta che A è vera.
L'equivalenza di 2 affermazioni è lo stesso che richiedere che ciascuna delle 2 implichi l'altra e quindi 2 affermazioni sono equivalenti se sono sempre entrambe vere o entrambe false. 

I simboli che verranno usati per rappresentare queste operazioni sono vari:

\begin{itemize}
\item
La congiunzione è rappresentata spesso da \&\& (come nel C++), ma spesso viene usata semplicemente la virgola per evitare di appesantire troppo le formule.
Quindi verrà scritto (A \&\& B) oppure (A, B).

\item
La disgiunzione viene rappresentata con la doppia sbarra verticale $\|$, come nel C++, e anche per la negazione useremo il punto esclamativo ! come nel C++.
Quindi scriveremo A $\|$ B e !A.

\item
Si rappresenta l'implicazione con $\Rightarrow$, mentre l'equivalenza la si rappresenta con $\Leftrightarrow$ (è la doppia implicazione!), ma anche con \textbf{se e solo se}, spesso abbreviato in \textbf{sse}.

\end{itemize}

Avremo bisogno di considerare insiemi \textit{S} di oggetti e di affermare che tutti gli elementi di \textit{S} soddisfano una certa proprietà \textit{P(·)}.
Questa affermazione è descritta dalla seguente formula:
\textit{$\forall$a $\in$ S,P(a)}.
Invece \textit{$\exists$a $\in$ S,P(a)} significa che esiste (almeno) un \textit{a $\in$ S} tale che valga \textit{P(a)}.
Vale la pena di osservare che qualora
\textit{S = $\emptyset$}, allora \textit{$\forall$a $\in$ S,P(a)} è vero perché richiedere che per tutti gli elementi di \textit{S} (cioè nessuno) valga \textit{P(·)} è banalmente vero.
Al contrario, se \textit{S = $\emptyset$}, allora \textit{$\exists$a $\in$ S,P(a)} è, altrettanto banalmente, falso perché essa richiede che ci sia qualche elemento \textit{a $\in$ S} per cui \textit{P(a)} sia vero.

Spesso verranno considerati gli insiemi composti da tutti gli interi tra 2 estremi \textit{i} e \textit{j}.
Un tale insieme è rappresentato con \textit{[i..j]}.
Qualora \textit{i > j}, \textit{[i..j]} rappresenta l'insieme vuoto.
In molti casi \textit{[i..j]} rappresenta indici di un array \textit{A} e con \textit{A[i..j]} si indica la sequenza degli elementi di \textit{A} dalla posizione \textit{i} alla \textit{j}.

\noindent Se \textit{[i..j]} è tale che \textit{i > j}, allora \textit{A[i..j]} rappresenta la sequenza vuota di elementi dell'array \textit{A}. 